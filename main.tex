\documentclass[UTF8,a4paper,12pt]{ctexart}
\usepackage{ctex}
\usepackage{amsmath}
\numberwithin{equation}{section}
\allowdisplaybreaks[4]       %多行公式中换页
\usepackage{array}
\usepackage[font=small,font=bf,labelsep=none]{caption}
\usepackage{amssymb}
\usepackage{tikz}
\usepackage{amsthm}
\usepackage{mathrsfs}
\usepackage{enumitem}
\usepackage{dutchcal}
\usepackage{multirow}
\usepackage{color}
\usepackage{titletoc}
\usepackage{graphicx}    %插入图片
\usepackage{times}
\usepackage{mathptmx}
\usepackage{fancyhdr} %页眉页脚
% \usepackage[figuresleft]{rotating}
\DeclareMathOperator*{\argmax}{argmax}
\DeclareMathOperator*{\argmin}{argmin}
\pagestyle{fancy}
\fancyhf{}
\fancyfoot[C]{\thepage}
\usepackage{setspace}
\setlength{\baselineskip}{20pt}
\newcommand*{\circled}[1]{\lower.7ex\hbox{\tikz\draw (0pt, 0pt)%
    circle (.5em) node {\makebox[1em][c]{\small #1}};}}
\usepackage{hyperref}  %目录
\hypersetup{colorlinks=true,linkcolor=black}
\renewcommand {\thefigure} {\thesection{}-\arabic{figure}}%设定图片的编号。这样设置的实现效果为图1-1
\renewcommand {\thetable} {\thesection{}-\arabic{table}}
\usepackage{caption}
\captionsetup{font={small},labelsep=quad}%文字5号,之间空一个汉字符位。
\captionsetup[table]{font={bf}} %表格表号与表题加粗
\usepackage{appendix}
\usepackage{tocloft} 
\renewcommand{\cftsecleader}{\cftdotfill{\cftdotsep}} %为目录中section补上引导点
\usepackage{titletoc}
\titlecontents{section}[0pt]{\addvspace{6pt}\filright\bf}%
               {\contentspush{\thecontentslabel \quad}}%
               {}{\titlerule*[8pt]{.}\contentspage}
\makeatletter %双线页眉
\def\headrule{{\if@fancyplain\let\headrulewidth\plainheadrulewidth\fi%
\hrule\@height 1.5pt \@width\headwidth\vskip1.5pt%上面线为1pt粗
\hrule\@height 0.5pt\@width\headwidth  %下面0.5pt粗
\vskip-2\headrulewidth\vskip-1pt}      %两条线的距离1pt
  \vspace{6mm}}     %双线与下面正文之间的垂直间距
\makeatother
\setmainfont{Times New Roman}
\CTEXsetup[format={\heiti \zihao{3} \bfseries \center}]{section}
\CTEXsetup[number={第\chinese{section}章}]{section} 
\usepackage[explicit]{titlesec}
\titlespacing*{\section}{0pt}{24pt plus .24pt minus .24pt}{18pt plus .0ex}

\usepackage[backend=biber,style=gb7714-2015]{biblatex}
\addbibresource[location=local]{papers.bib}%biblatex宏包的参考文献数据源加载方式
% \setlength{\bibitemsep}{0pt}
% 导入参考文献数据库
% \addbibresource{papers.bib}
\begin{document}

\thispagestyle{empty}

\renewcommand{\headrulewidth}{0pt}
\begin{figure}[htb] 
 \center{\includegraphics[width=5cm]  {images/fig1.png}} 
 \end{figure}

\begin{center}
\songti \zihao{-2} 上海交通大学学位论文
\end{center}
%该页为中文扉页。无需页眉页脚,纸质论文应装订在右侧
~\\
\begin{center}
\songti \zihao{2} \textbf{标题}
\end{center}
%中文论文标题,1行或2行,宋体,加粗,二号,居中。论文题目不得超过36个汉字
~\\
~\\
~\\
~\\
\begin{center}
\heiti \zihao{4}
\begin{tabular}{l}
\textbf{姓\quad 名:名字}\\
\textbf{学\quad 号:111122223333}\\
\textbf{导\quad 师:老板}\\
\textbf{学\quad 院: xx学院}\\
\textbf{学科/专业名称:xx技术}\\
\textbf{申请学位层次:x士}\\
\end{tabular}
\end{center}
~\\
\begin{center}
\songti \zihao{4} \textbf{xxxx年x月}
\end{center}
\newpage
\thispagestyle{empty}
~\\
\begin{center}
\zihao{4}
\textbf{
A Dissertation Submitted to \\
Shanghai Jiao Tong University for Master Degree}
\end{center}
~\\
\begin{center}
\zihao{-2}\textbf{Title tile
}
\end{center}
%英文论文标题:大写,Times New Roman,加粗,14 points,居中
~\\
~\\
~\\
\begin{center}
\zihao{3} 
Author: name\\
Supervisor:  teacher
\end{center}
~\\
~\\
~\\
\begin{center}
\zihao{3} 
School of xxxx\\
Shanghai Jiao Tong University \\
Shanghai, P.R.China \\
March 31th, 20xx  
\end{center}
\newpage
\thispagestyle{empty}
\begin{center}
\heiti \zihao{3}\textbf{
上海交通大学\\
学位论文原创性声明}
\end{center}

\zihao{-4}
本人郑重声明:所呈交的学位论文,是本人在导师的指导下,独立进行研究工作所取得的成果。除文中已经注明引用的内容外,本论文不包含任何其他个人或集体已经发表或撰写过的作品成果。对本文的研究做出重要贡献的个人和集体,均已在文中以明确方式标明。本人完全知晓本声明的法律后果由本人承担。

\begin{flushright}
\begin{tabular}{l}
\zihao{4}
学位论文作者签名:\hspace{20mm}\qquad\\
\zihao{4}
日期:\qquad 年\qquad 月\qquad 日
\end{tabular}
\end{flushright}

~\\
\begin{center}
\heiti \zihao{3}\textbf{
上海交通大学\\
学位论文使用授权书}
\end{center}

本人同意学校保留并向国家有关部门或机构送交论文的复印件和电子版,允许论文被查阅和借阅。\\
本学位论文属于 :\par
□公开论文\par
□内部论文,保密□1年/□2年/□3年,过保密期后适用本授权书。\par
□秘密论文,保密\_\_\_年(不超过10年),过保密期后适用本授权书。\par
□机密论文,保密\_\_\_年(不超过20年),过保密期后适用本授权书。\par
(请在以上方框内选择打“√”)\\

\begin{flushright}
\zihao{4}
\begin{tabular}{l l}
学位论文作者签名:\hspace{10mm}\qquad \hspace{100mm}&指导教师签名:\qquad\\
日期:\qquad 年\qquad 月\qquad 日 &日期:\qquad 年\qquad 月\qquad 日\\
\end{tabular}
\end{flushright}
\newpage
\pagenumbering{Roman}
\fancyhead[LH]{上海交通大学学位论文}
\fancyhead[RH]{摘\quad 要}

\phantomsection
\addcontentsline{toc}{section}{摘\quad 要}

\section*{摘\quad 要}
%摘要:二字间空一格,黑体16磅加粗居中,单倍行距,段前24磅,段后18磅。

\hspace{8mm}摘要\\
~\\
\textbf{关键词}:关键词1\\
%关键字:宋体12磅,行距20磅,段前段后0磅,关键字之间用逗号隔开,关键词三个字加粗。


\newpage
\phantomsection
\addcontentsline{toc}{section}{ABSTRACT}
\section*{ABSTRACT}
%ABSTRCT:Arial 16磅加粗居中,单倍行距,段前24磅,段后18磅

\hspace{8mm}abstract.
\\
%英文摘要内容:Times New Roman 12磅,行距20磅段前段后0磅
~\\ 
\textbf{Key words}: k1
%Keywords:Times New Roman 12磅,行距20磅, “key words” 两词加粗
\newpage
\renewcommand\contentsname{\textbf{目\quad 录}}
\thispagestyle{fancy}
\fancyhead [RO, LE] {\normalsize{\songti 目\quad 录}}
\fancyhead [LO, RE] {\normalsize{\songti 上海交通大学学位论文}}
\begin{center}
    \tableofcontents
\end{center}
\newpage
\pagenumbering{arabic}
\fancyhead[LH]{上海交通大学学位论文}
\fancyhead[RH]{\ref{sec:introduction}\quad 绪论}
\section{绪论}\label{sec:introduction}
\subsection{引言} % 

本文介绍了xxx。

\subsection{本文研究意义}\label{subsec:introduction_meaning}
refer to paper~\cite{paper}.


\subsection{本章小结}
本章初步介绍了本文的主要研究问题
\newpage
\fancyhead[LH]{上海交通大学学位论文}
\fancyhead[RH]{~\ref{sec:conclusion}\quad 全文总结}
\section{全文总结}\label{sec:conclusion}

\subsection{主要结论}

本文主要解决了问题。

\begin{itemize}
    \item 一个点
\end{itemize}
\newpage
\fancyhead[LH]{上海交通大学学位论文}
\fancyhead[RH]{参考文献}
\phantomsection
\addcontentsline{toc}{section}{参考文献}
\renewcommand\refname{参考文献}
\printbibliography[heading=bibliography,title=参考文献]

\newpage
\fancyhead[LH]{上海交通大学学位论文}
\fancyhead[RH]{附录}
\appendix
\newpage
\fancyhead[LH]{上海交通大学学位论文}
\fancyhead[RH]{附录 ~\ref{subsec:appendix_1}}
\section{f1}\label{subsec:appendix_1}

section


\newpage
\fancyhead[LH]{上海交通大学学位论文}
\fancyhead[RH]{学术论文和科研成果目录}
\phantomsection
\addcontentsline{toc}{section}{攻读学位期间学术论文和科研成果目录}
\section*{攻读学位期间学术论文和科研成果目录}
\begin{enumerate}[label = {[\textbf{\arabic*}]}]
    \item {One published paper}
    \item {Another published paper}
    
\end{enumerate}

\newpage
\fancyhead[LH]{上海交通大学学位论文}
\fancyhead[RH]{致\qquad 谢}

\phantomsection
\addcontentsline{toc}{section}{致\qquad 谢}

\section*{致\qquad 谢}

\hspace{8mm} 

感谢上海交通大学
\end{document} 